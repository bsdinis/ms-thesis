% #############################################################################
% This is Appendix A
% !TEX root = ../main.tex
% #############################################################################
\chapter{Correctness Proofs}\label{chapter:appendixA}

In this appendix, we provide the proofs of correctness for the
protocols presented in the paper. The appendix is organized as
follows:~\ref{app:quorum} presents a proof of an auxiliary
theorem regarding \ac{RR} quorum systems;~\ref{app:register}
contains a proof of correctness of the distributed register
protocol; and~\ref{app:smr} a proof of correctness of the state
machine replication protocol.

\section{\ac{RR} Quorum Systems}\label{app:quorum}

From the parametrization of \ac{RR} quorum systems derived
in Section~\ref{ssec:parameters}, we can prove a theorem that
characterizes the properties we obtain from these systems and
their quorum sizes.

\begin{theorem}\label{thrmQ}
  The quorum system defined by arbitrary subsets with the
    cardinalities in Equations
    (\ref{eq:fresh},\ref{eq:ReplSize},\ref{eq:rq_size}) is a
    Dissemination Quorum System~\cite{Malkhi:Reiter:BQS:98}.
\end{theorem}

\begin{dem}
The D-Consistency property from Definition 5.1
in~\cite{Malkhi:Reiter:BQS:98} follows from the fact that
constraint in Equation~(\ref{eq:inters}) was preserved
throughout the derivation. In particular, in the final quorum
sizes we consider, we have that the read and a write quorum
intersect in the following minimum number of replicas:

\begin{align*}
  W_Q + R_Q - N &=\\
  M_R + \Delta_W + R_Q - M_R - F - \Delta_W - \Delta_N  &=\\
  R_Q - F - \Delta_N  &=\\
  F + \min(s, M_R) + \Delta_N + \Delta_R - F - \Delta_N  &=\\
  \min(s, M_R) + \Delta_R  &
\end{align*}

\noindent which obeys the D-Consistency property because it
contains at least one non-rolled back replica ($\Delta_R >
0$), which directly implies that such a replica does not
belong to any set $B \in \cal{B}$ according to the definition
in~\cite{Malkhi:Reiter:BQS:98}.

Similarly, the D-Availability property follows from the fact that
the liveness equations were preserved throughout the
derivation. In particular, in the final quorum sizes we
consider, we have that:

\begin{align*}
    N-W_Q &=\\
    M_R + F + \Delta_W + \Delta_N - M_R - \Delta_W &=\\
    F + \Delta_N&
\end{align*}

and

\begin{align*}
    N-R_Q &=\\
    M_R + F + \Delta_W + \Delta_N - F - \min(s, M_R) - \Delta_N - \Delta_R &=\\
    M_R + \Delta_W - \Delta_R - \min(s, M_R) &\ge f'
\end{align*}
\noindent which holds given the read liveness conditions from
Equation~\ref{eq:DynRavail}.

\hfill\ensuremath{\Box}\vspace{2em}

\end{dem}

\section{Distributed Register}\label{app:register}

In this section we prove the safety of the distributed register
protocol, namely that it obeys linearizable semantics.

%% \rrnote{Algorithm highlights ---
%%     two phase read [R1, R2] + stabilize in bg /
%%     two phase write [W1, W2] + stabilize in bg /
%%     three phase cond-write: tentative-write/write/stabilize (in fg) [C1,C2,C3] /
%%     write will fail if W1 sees pre-written cond-write in progress /
%%     cond-write will abort in C1 and return fail if it sees a concurrent write or cond-write in progress /
%%     stabilize conditions: $W_Q$ of R2 ; $W_Q$ of W2 ; $W_Q$ of C2
%% }

\subsection{Specification}

The specification is simply stated as linearizability of an
object that supports read and write operations.
Linearizability states that there exists a sequential history (or
linearization) of the operations that took place in a history
of the execution of the system, such that the linearization leads
to the same outputs according to a sequential specification, and is
compatible with the real time precedence of the operations. More
precisely, this can be stated as:

\begin{itemize}
    \item[] \textbf{R-L1}: there is a total order $<$ of
        operations (reads and writes), consistent with the
        real-time invocation/reply order (meaning that if $op_1$
        returns before $op_2$ is invoked then $op_1 < op_2$)
    \item[] \textbf{R-L2}: reads return the value written by the
        most recent write according to that order.
\end{itemize}

Note that a formal proof that these properties imply linearizable semantics
can be found in~\cite{nancy-book}.

\subsection{Proof}

\paragraph{R-L1}
For proving the existence of this order, we construct the
the total order $<$ as the
lexicographic order of timestamps $<sequence number, id>$.
However, this is not yet a total order, since reads will have the same
timestamp as the operation that created the value that was returned. We
thus add the additional constraint that reads are ordered immediately
after the operation that wrote the value that was read. When several
read opearations return the same value, then we order the
reads among themselves in any total order that is compatible with the
real-time order of invocation, i.e., with the constraint that
if $read_1$ returns before $read_2$ is invoked then $read_1 < read_2$.

Given this construction, we now need to proof that this order is
consistent with the real-time precedence order, i.e., that if
operation $op_2$ is invoked after operation $op_1$ returns, then
$op_1 < op_2$.

If $op_1$ is a read, then it only concludes after either writing
back the return value to a write quorum, or reading unanimously
from a write quorum, or being signaled by the stable flag that
the value was previously present at a write quorum. Therefore,
according to Theorem~\ref{thrmQ},  a
subsequent first phase of a read or write
($op_2$) will see that timestamp at its read
quorum (or, given that timestamps grow monotonically at each replica, a higher one), and will therefore be serialized at a subsequent point in
the total order either by picking a higher timestamp for writes, or by returning the same or a higher timestamp
for reads. (In case $op_2$ is a read that returns the same timestamp,
then it obeys the required property directly by the way that $<$ is
constructed for that particular case.)

This argument also applies when $op_1$ is a write, since the second phase of each operation
also reaches a write quorum.

\paragraph{R-L2}
This follows directly from the way that the total order $<$ is
constructed. In particular, reads are serialized after the write
with the same timestamp, which is the write
that wrote the value that is returned by the
read.
%% The only problem would be if the timestamp corresponded
%% to a write that has failed, since that write
%% must not be seen. However, we prove that this is not
%% possible. Returning a failed write: writes only return fail
%% if they fail in the first phase, but in this phase they only
%% read without leaving any state at the replicas. Therefore
%% there is no possibility of a subsequent read returning a
%% failed write.

\section{State Machine Replication}\label{app:smr}

In this section, we prove the safety of the state machine
replication protocol, again using linearizable semantics as the
notion of correctness.

\subsection{Specification}

The specification is stated as linearizability of an object that
supports read and update operations. By the definition of
linearizability, there exists a sequential history
(linearization) of operations that took place in the execution of
the system. This sequential history must be compatible with the
real time order of operations and be equivalent (i.e.,\ lead to
the same outputs as) a sequential specification of the state machine.
More precisely,
it can be stated as:

\begin{enumerate}
    \item [] \textbf{SM-L1}: there is a total order $<$ of state machine operations
        (reads and updates), consistent with the real-time
        invocation/reply order (i.e.,\ if $op_1$ returns before
        $op_2$ is inkoved, then $op_1 < op_2$);

      \item [] \textbf{SM-L2}: state machine operations return the value that results from the
        sequential execution of the sequence of preceding operations
        according to that order.
\end{enumerate}

As in the previous proof, the same helper theorem~\cite{nancy-book} can
be used to show that these properties imply linearizable semantics, since
it can apply to any atomic object, including a state machine.


\subsection{Proof}


\paragraph{SM-L1}
For proving the existence of this order, we construct the
the total order $<$ as the lexicographic order of slot numbers.
However, this is not yet a total order, since reads will have the same
slot number as the operation that created the value that was returned. We
thus add the additional constraint that reads are ordered immediately
after the operation that wrote the value that was read. When several
read opearations return the same value, then we order the
reads among themselves in any total order that is compatible with the
real-time order of invocation, i.e., with the constraint that
if $read_1$ returns before $read_2$ is invoked then $read_1 < read_2$.

Given this construction, we now need to proof that this order is
consistent with the real-time precedence order, i.e., that if
operation $op_2$ is invoked after operation $op_1$ returns, then
$op_1 < op_2$.

If both operations are updates, this follows from the
construction of the protocol, wherein the leader chooses the slot
position for the update and then, by collecting a super quorum of
accepts, guarantees that no subsequent update will be assigned to
that position (or a lower one), and no subsequent read will see a
lower slot number. Furthermore, for the case that the first
operation is a read, it will require a unanimous read quorum, and
additionally the replicas in this read quorum have not sent out
accept messages to other replicas. Given these protocol
mechanisms, the intersection property implied by
Theorem~\ref{thrmQ} (read quorums intersect super quorums at a
correct replica) ensure that there is no update committed to a
slot number higher than the one returned in the log of any
replica.

\paragraph{SM-L2}
This follows from the way that the total order $<$ is
constructed. In particular, (1) reads are serialized after the
update with the same slot number, which is the latest update
in the sequence of updates that leads to a state of the state machine
that corresponds to the value that is returned by the
operation, and (2) updates are serialized according to their
slot number order, with is also the order in which replicas execute those updates,
leading to outputs that are the same as the sequential specification
of the state machine.
