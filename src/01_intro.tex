\fancychapter{Introduction}\label{chap:intro}
\cleardoublepage{}

\if 0
\bsd{\begin{enumerate}
    \item State of the Art: replicated systems with persistent
        state are treated akin to volatile state replicated
        systems, with the additional option of a node recovering
        from local state after restart;
    \item In certain security sensitive contexts, namely TEEs.
        this does not work. Nodes have limited control over
        persistent state, especially across restarts.
    \item To recover safely from persistent state one must use a
        stronger fault model. Currently, the only adequate model
        is BFT. This is far too strong, considering that nodes
        have correct and trusted execution.
    \item To bridge this gap, we develop the RR model, where
        a node's persistent state can be rolled back to a
        previous valid version. This way, TEEs can recover from
        their local persistent state.
    \item Taking a step back, this can be useful to extract
        performance from other replicated systems. In particular,
        by being more relaxed with syncronizing the volatile
        and persistent states, we can batch more aggressively.
        This would be otherwise unsafe, as a crash could rollback
        a replica.
    \item To evaluate our approach, we developed two systems,
        TEEMS and R2-S2. (Description of the systems and
        experimental results)
\end{enumerate}}
\fi

Replication is a standard technique in distributed systems, which adds
fault tolerance to a service implemented by an individual networked
node. To ensure that the replicated service appears to clients like
its single-node equivalent except for higher availability, a
replication protocol coordinates the replicated
nodes. For instance, to replicate a \ac{KVS} (i.e.,\ a
storage system which offers a simple get/put interface) one can
use the \ac{ABD}~\cite{abd} protocol. For more complex
applications that can be modeled using a deterministic state
machine, a \ac{SMR} protocol like Paxos~\cite{paxos} may be used.

Durability, i.e.\ not losing data even in the event of faults, is a key property of storage
systems. For replicated storage systems, adding local persistence
to the nodes that make up the system is a common practice. This
allows replicas to recover from crashes (in which they lose their
volatile state) from their local persistent state, which in turn makes the
system resilient to catastrophic full system
shutdowns, a property considered as being of paramount importance
in real-world deployment of replicated systems~\cite{bolosky:paxos}. From a
formal point of view, however, replicated systems that persist
the state of each replica durably are commonly treated in a
very similar fashion to others that store their state solely in
volatile memory --- the persistent aspect is considered trivial.
Indeed, when the persistent aspect of nodes is explicitely
highlighted, protocols only add small extensions that describe
when and what data nodes need to persist so that, in the event of
a restart, the node can operate as if nothing happened.

This assumption that persisting the local state of
replicated nodes is trivial does not hold in certain security sensitive cases, in
particular in the context of \acp{TEE}. \acp{TEE} are
applications running in a special hardware-protected environment,
where the CPU shields the application, guaranteeing the integrity
and confidentiality of the code execution and the associated
volatile memory, in a way that it can be remotely attested that
the hardware protections were correctly set up. Consequently, they
have become an attractive solution for system administrators who
make deployments in infrastructure that they do not directly
control (e.g., the public cloud). \acp{TEE} run in an highly
adversarial attacker model, where the owner and operator of the
hardware is not trusted. As such, although nodes can keep
integrity information (i.e., a cryptographic hash) in memory and encrypt their persistent state,
to ensure that it cannot be arbitrarily tampered with, their
control over the persistent storage is quite limited.
Particularly, when a restart happens and the volatile integrity
information is lost, the \ac{TEE} may suffer a \emph{rollback attack}, where
the attacker replaces the persistent state with a stale but
correct version of the state. Simply applying current replicated
protocols would pose a correctness and security problem.

Despite these challenges, networked services implemented inside \acp{TEE} like
ARM TrustZone~\cite{armTZ}, AMD SEV-SNP~\cite{amdsev,
amdsev-snp}, Intel SGX~\cite{intelsgx}, or the future Intel
TDX~\cite{inteltdx} and ARM CCA~\cite{arm-cca} can similarly
benefit from replication.  Replicated \ac{TEE}-based systems aim
to combine the confidentiality, integrity, and remote attestation
of \acp{TEE} with replication to ensure high availability.
Examples include permissioned blockchains~\cite{teechain},
monotonic counters for ensuring state freshness~\cite{rote},
in-memory \acp{KVS}~\cite{avocado-atc21}, as well as
broadcast and common random number generator
primitives~\cite{p2p-sgx}.

A key design decision in all replicated systems is the choice of fault
model underlying the replication protocol. This model captures the set
of faults that can affect an individual replica.  For instance, in the
{\em crash-fault model}, faulty nodes are assumed to execute correctly
until they crash at an arbitrary point in their execution, when they
seize to perform further steps until they restart. In the {\em
  Byzantine fault model}, faulty nodes may perform arbitrary
actions. The choice of fault model affects the complexity, overhead,
and tolerance threshold of a replication protocol, and is important
for both performance and security. A fault model that is too
pessimistic may lead to unnecessary safeguards, which can impact both
performance and the cost of replication. Conversely, a fault model
that is too optimistic may fail to account for all faults that can
occur, which then leads to broken assumptions and loss of correctness
or security of the replicated system.

In the case of \ac{TEE}-based replication, the choice of existing systems
that store replica state persistently~\cite{teechain,rote} was to
employ \ac{BFT} replication. This may seem
necessary given that rollback of persistent state is a behavior not
covered by the crash-fault model.  However, the Byzantine model is
much stronger than necessary for this setting. Intuitively, this
is because it assumes arbitrary behavior of
faulty components, thus not taking into account the integrity
guarantees for the code running inside \acp{TEE}.

To fill this gap in distributed replication research, we
introduce the \ac{RR} fault model, which
%is designed for replicated \ac{TEE}-isolated services and
captures precisely the set of faults that \acp{TEE} can suffer according to
their specification.  The main insight behind \ac{RR} is that,
apart from crash faults, whenever a \ac{TEE} restarts, the fault model
allows for a rollback of externally stored state to an earlier
version.  Such rollback events are possible because \acp{TEE} lack general
and high-performance means of ensuring the freshness of externally
stored state.  During an execution, however, the integrity of the
internal, volatile state of a \ac{TEE} is ensured by hardware, and the
integrity of any externally stored state can be ensured by standard
cryptographic means.

The \ac{RR} fault model occupies a middle ground between \ac{CFT}
and \ac{BFT} models. As we will show, unlike CFT it provides
security for replicated \acp{TEE} by tolerating state roll-back at a cost
that is similar to \ac{CFT} and better than \ac{BFT}. Specifically, it is able
to substantially reduce the required communication, particularly for
read operations, in the common case where restarts are
infrequent.

Stepping back, we make the observation that the \ac{RR} fault
model can be useful, albeit for different reasons, in the more
general case of persistent replicated systems within the crash
model. A key factor for performance in this systems are the slow
accesses to untrusted storage, which must occur in the critical
path to ensure the correctness of the system. By batching
multiple requests together, it is possible to reduce this
overhead, increasing throughput at the expense of the additional
latency required to form the batch. Leveraging the \ac{RR} model,
it is possible to form larger batches, replying to
certain requests before synchronizing to persistent storage. This
would otherwise be unsafe, as a crash before the synchronization
would constitute an effective rollback, which the \ac{RR} model
captures.

To demonstrate the potential of the \ac{RR} model, we apply it to
two types of replication protocols, namely the \ac{ABD}
 read/write replication protocol and the Paxos \ac{SMR}
protocol, showing that this adaptation is relatively straightforward.
We then use these protocols to devise two systems: \ac{TEEMS} and
\ac{R2-S2}. \ac{TEEMS}, a replicated metadata service,
is a relevant contribution in its own right, since
it enables Cloud storage with the same strong confidentiality and
integrity guarantees provided by \acp{TEE}. while also ensuring freshness
of data. \ac{TEEMS} is used with one or more untrusted cloud storage
services. \ac{TEEMS} maintains metadata for each data item (including the
encryption key, authentication code, latest version number and
policy), while ensuring strong security and enabling concurrent
sharing of data. \ac{R2-S2} is a replicated \ac{KVS} built
using regular nodes (i.e., without TEEs or other security
mechanisms) that leverages the \ac{RR} model with the objective
of extracting higher throughput while retaining the strong reliability expected of a persistent
replicated system.

\section{Contributions}

In the \ac{TEE} context, we implemented the read/write
register and the \ac{SMR} protocols in three fault models:
\ac{CFT}, \ac{RR} and \ac{BFT}, and a prototype for \ac{TEEMS},
all using Intel SGX. In the storage context, we implemented
\ac{R2-S2} as a replication layer on top of a single node key
value store. We evaluated our prototypes using both
microbenchmarks and real-world workloads, namely
\ac{YCSB}~\cite{ycsb}.

Our contributions include:
\begin{enumerate}
    \item The \ac{RR} fault model, which is the first to capture
        precisely the fault behavior of \acp{TEE} that rely on external storage;
    \item A derivation of quorum sizes for replication protocols
        in the \ac{RR} model;
    \item Principles for the adaptation of \ac{CFT} protocols to
        the \ac{RR} model;
    \item Two protocols for this model, implementing read-write
        storage and \ac{SMR};
    \item The \ac{TEEMS} generic metadata service, and its integration with
        untrusted cloud storage services to give clients the ability to share
        and concurrently access persistent data with strong confidentiality,
        integrity, and freshness;
    \item The \ac{R2-S2} distributed \ac{KVS}, which
        enables higher throughput from \acp{KVS} backed by
        persistent state;
    \item An extensive experimental evaluation of our protocols
        in the \ac{TEE} setting, with a detailed comparison with
        unsafe \ac{CFT} and expensive \ac{BFT} solutions;
    \item An preliminary experimental evaluation of the \ac{R2-S2}
        system.
\end{enumerate}


% #############################################################################
\section{Organization of the Document}

The rest of the thesis is organized as follows:
\Cref{chap:related} covers related work. In \cref{chap:model} we
formalize the \ac{RR} model and show the derivation of
\ac{RR}-based quorum systems and protocols. \Cref{chap:tee}
describes the usage of \acp{TEE} with \ac{RR}, with the
accompanying design and evaluation of \ac{TEEMS}. In
\cref{chap:storage} we present the design of
\ac{R2-S2} and its experimental evaluation.
\Cref{chap:conclusion} summarizes our work and presents possible
avenues for future work.

% #############################################################################

\section{Publication}

A portion of this thesis, namely most of the material covered in
Chapters~\ref{chap:model} and~\ref{chap:tee}, is featured in a
paper conditionally accepted (decision of ``minor revision'') to
the Network and Distributed System Security (NDSS) Symposium
2023.

% #############################################################################

\if 0 % paper intro

Replication is a standard technique in distributed systems, which adds
fault tolerance to a service implemented by an individual networked
node. To ensure that the replicated service appears to clients like
its single-node equivalent except for higher availability, a
replication protocol coordinates the replicated
nodes. For instance, a state machine replication (SMR) protocol like
Paxos~\cite{paxos} may be used for this purpose.

Networked services implemented inside a trusted execution environment
(TEE) like ARM TrustZone~\cite{armTZ}, AMD SEV-SNP~\cite{amdsev,
  amdsev-snp}, Intel SGX~\cite{intelsgx}, or the future Intel
TDX~\cite{inteltdx} and ARM CCA~\cite{arm-cca} can similarly benefit
from replication.  Replicated \ac{TEE}-based systems aim to combine the
confidentiality, integrity, and remote attestation of \acp{TEE} with
replication to ensure high availability.
%and resilience to rollback attacks that revert the externally stored,
%persistent state of a TEE~\cite{memoir}
%\bsd{the reference to rollback attacks early on seems out of place,
%  since this is a gap in most replicated \acp{TEE}. Moreover, memoir is not
%  a replicated \ac{TEE} system}.
Examples include permissioned blockchains~\cite{teechain}, monotonic
counters for ensuring state freshness~\cite{rote}, in-memory key-value
stores~\cite{avocado-atc21}, as well as broadcast and common random
number generator primitives~\cite{p2p-sgx}.

A key design decision in all replicated systems is the choice of fault
model underlying the replication protocol. This model captures the set
of faults that can affect an individual replica.  For instance, in the
{\em crash-fault model}, faulty nodes are assumed to execute correctly
until they crash at an arbitrary point in their execution, when they
seize to perform further steps until they restart. In the {\em
  Byzantine fault model}, faulty nodes may perform arbitrary
actions. The choice of fault model affects the complexity, overhead,
and tolerance threshold of a replication protocol, and is important
for both performance and security. A fault model that is too
pessimistic may lead to unnecessary safeguards, which can impact both
performance and the cost of replication. Conversely, a fault model
that is too optimistic may fail to account for all faults that can
occur, which then leads to broken assumptions and loss of correctness
or security of the replicated system.

In the case of \ac{TEE}-based replication, the choice of existing systems
that store replica state persistently~\cite{teechain,rote} was to
employ Byzantine fault tolerant (BFT) replication. This may seem
necessary given that rollback of persistent state is a behavior not
covered by the crash-fault model.  However, the Byzantine model is
much stronger than necessary for this setting. Intuitively, this
is because it assumes arbitrary behavior of
faulty components, thus not taking into account the integrity
guarantees for the code running inside \acp{TEE}.

In this paper, we fill a gap in distributed replication research by
introducing the restart-rollback (\faultmodel) fault model, which
%is designed for replicated \ac{TEE}-isolated services and
captures precisely the set of faults that \acp{TEE} can suffer according to
their specification.  The main insight behind \faultmodel\ is that,
apart from crash faults, whenever a \ac{TEE} restarts, the fault model
allows for a rollback of externally stored state to an earlier
version.  Such rollback events are possible because \acp{TEE} lack general
and high-performance means of ensuring the freshness of externally
stored state.  During an execution, however, the integrity of the
internal, volatile state of a \ac{TEE} is ensured by hardware, and the
integrity of any externally stored state can be ensured by standard
cryptographic means.

The \faultmodel\ fault model occupies a middle ground between crash
fault tolerance (CFT) and BFT. As we will show, unlike CFT it provides
security for replicated \acp{TEE} by tolerating state roll-back at a cost
that is similar to CFT and better than BFT. Specifically, it is able
to substantially reduce the required communication, particularly for
read operations, in the common case where restarts are
infrequent.

To demonstrate the potential of the \faultmodel\ model, we apply it to
two types of replication protocols, namely the Attiya, Bar-Noy, Dolev
(ABD) read/write replication protocol and the Paxos
state machine replication (SMR)
protocol, showing that this adaptation is relatively straightforward.
%; and, we apply each protocol to a specific application scenario.
We then use these protocols to devise a replicated metadata service
called \sys, which is a relevant contribution in its own right, since
it enables Cloud storage with the same strong confidentiality and
integrity guarantees provided by \acp{TEE}. while also ensuring freshness
of data.
% and the ability to securely share data between clients by
% enforcing access policies. To this end, \sys\ provides a
% generic, scalable,
%\sys\ is a trustworthy metadata service hosted in a replicated set of
%TEEs, which may themselves be hosted by a diverse set of cloud
%providers for increased resilience.
%This service provides a semantically rich interface with
%persistent updates and read-modify-write (RMW) operations.
\sys\ is used with one or more untrusted cloud storage
services. \sys\ maintains metadata for each data item (including the
encryption key, authentication code, latest version number and
policy), while ensuring strong security and enabling concurrent
sharing of data.

%and updates this metadata and the respective data atomically using the simple read/write protocol for the \faultmodel\ model.\bsd{need to rewrite this, present that it uses both protocols}

%The second case study is inspired by permissioned blockchains, and
%their ability to improve performance by offloading transactions to a
%payment channel running a more efficient protocol, and then settling
%the final balance to the main blockchain. To support this, we built a
%state machine replication protocol based on the \faultmodel\ model,
%thus allowing the more complex operations of these payment channels to
%be supported in a resilient way.

We implemented both the read/write and SMR replication protocols for
three different fault models (\faultmodel, crash, and Byzantine), and
we also implemented \sys\ and used it to implement a new class of
secure cloud storage services. We evaluated the protocols using
microbenchmarks and the storage system using both microbenchmarks and
YCSB.  Our experimental results show that the protocols for the
\faultmodel\ model perform significantly better than their BFT
counterparts, and, compared to CFT, they perform identically while
additionally tolerating rollback attacks.
Furthermore, \sys\ offers secure, shareable storage at modest overhead.

Our contributions include:
%\begin{enumerate}
%\item
 (1) The \faultmodel\ fault model, which is the first to capture
precisely the fault behavior of \acp{TEE} that rely on external storage;
%    \item
(2) a derivation of quorum sizes for replication protocols in the
\faultmodel\ model;
      %     \item
(3) principles for the adaptation of CFT protocols to
the \faultmodel\ model;
%\item
(4) two protocols for this model, implementing read-write storage and
SMR;
        %    \item
(5) The \sys\ generic metadata service, and its integration with
untrusted cloud storage services to give clients the ability to share
and concurrently access persistent data with strong confidentiality,
integrity, and freshness;
%\item
(6) An extensive experimental evaluation of our protocol, and of the
storage solutions based on \sys\ in different deployment scenarios.
%\end{enumerate}


The remainder of the paper is organized as follows.  We motivate
our work and provide some background \S\ref{sec:fault_model},
describe the \faultmodel\ model in \S\ref{sec:model}, present the
replication protocols in the new model in
\S\ref{sec:protocol}, and an example application built using
those protocols in \S\ref{sec:metadata}.  We describe our
prototype implementations in \S\ref{sec:impl}, evaluate them
experimentally in \S\ref{sec:eval}, survey related work in
\S\ref{sec:related_work}, and conclude in \S\ref{sec:conclusion}.





\fi
