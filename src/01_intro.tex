\fancychapter{Introduction}\label{chap:intro}
\cleardoublepage{}

\bsd{\begin{enumerate}
    \item State of the Art: replicated systems with persistent
        state are treated akin to volatile state replicated
        systems, with the additional option of a node recovering
        from local state after restart;
    \item In certain security sensitive contexts, namely TEEs,
        this does not work. Nodes have limited control over
        persistent state, especially across restarts.
    \item To recover safely from persistent state one must use a
        stronger fault model. Currently, the only adequate model
        is BFT. This is far too strong, considering that nodes
        have correct and trusted execution.
    \item To bridge this gap, we develop the RR model, where
        a node's persistent state can be rolled back to a
        previous valid version. This way, TEEs can recover from
        their local persistent state.
    \item Taking a step back, this can be useful to extract
        performance from other replicated systems. In particular,
        by being more relaxed with syncronizing the volatile
        and persistent states, we can batch more aggressively.
        This would be otherwise unsafe, as a crash could rollback
        a replica.
    \item To evaluate our approach, we developed two systems,
        TEEMS and R2-S2. (Description of the systems and
        experimental results)
\end{enumerate}}

% #############################################################################
\section{Organization of the Document}
\bsd{TODO}
\if 0
This thesis is is organized as follows: \Cref{chap:intro}
interdum vel, tristique ac, condimentum non, tellus.  In
\cref{chap:back} curabitur nulla purus, feugiat id, elementum in,
lobortis quis, pede.  In \cref{chap:architecture} consequat
ligula nec tortor. Integer eget sem. Ut vitae enim eu est
vehicula gravida.  \Cref{chap:implement} morbi egestas, urna non
consequat tempus, nunc arcu mollis enim, eu aliquam erat nulla
non nibh in \cref{chap:evaluation}.  \Cref{chap:conclusion}
suspendisse dolor nisl, ultrices at, eleifend vel, consequat at,
dolor.
\fi
