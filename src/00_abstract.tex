% #############################################################################
% Abstract Text
% !TEX root = ../main.tex
% #############################################################################
% reset acronyms
\acresetall
% use \noindent in firts paragraph

\vskip -.5cm
\noindent \acsp{TEE} ensure the confidentiality and integrity of
computations in hardware. Subject to the \acs{TEE}'s threat model,
the hardware shields computation from most externally induced
faults except crashes. As a result, \acs{CFT}
replication protocols should be sufficient when replicating
\acsp{TEE}. However, \acsp{TEE} do not offer
efficient means of ensuring the freshness of persistent state,
which can be rolled back to an earlier version. In this
dissertation, we propose the \ac{RR} fault model for replicating
\acsp{TEE}, which precisely captures the possible fault behaviors
of \acsp{TEE} with external state, making it possible to avoid
using expensive \acs{BFT} protocols. Then, we show that existing
\acs{CFT} replication protocols can be easily adapted to this fault model
with few changes, while retaining their original performance.
\vskip 0cm
To illustrate the usefulness of \acs{TEE} replication under
\acs{RR}, we built a replicated metadata service called
\acs{TEEMS}, which can be used to add \acs{TEE}-grade
confidentiality, integrity, and freshness to untrusted cloud
storage. Then, we showcase the generality of the \acs{RR} model,
by applying it to the context of replicated \acsp{KVS}. Current
replicated \acsp{KVS} make a tradeoff between latency, throughput
and durability, depending on whether they flush writes before
replying (which offers poor performance) or batch writes in
memory (possibly losing data if nodes crash before flushing). We
observe that the latter case effectively constitutes a rollback,
which is captured by \acs{RR}.
\vskip 0cm
We implemented the various protocols and systems, and our
evaluation shows performance that is comparable to \acs{CFT}
systems but with stronger guarantees, and significantly better
than \acs{BFT} systems.
