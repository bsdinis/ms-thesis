% #############################################################################
% RESUMO em Português
% !TEX root = ../main.tex
% #############################################################################
% use \noindent in firts paragraph
% reset acronyms
\acresetall
\noindent Ambientes de Execução Confiável (em inglês, TEEs)
garantem a confidencialidade e integridade de computações ao
nível do \textit{hardware}. Sobre o modelo adversarial dos TEEs, o
\textit{hardware} protege a computação da maioria das falhas
induzidas externamente, excepto falhas de \textit{crash}, em que
um TEE simplesmente vai abaixo (e.g., quando falta a
eletricidade). Contudo, TEEs, não possuem métodos eficientes de
garantir a frescura do seu estado persistente, que pode ser
revertido para uma versão anterior. Nesta dissertação, propomos o
modelo de faltas \ac{RR} para replicação de TEEs, que captura
exatamente os comportamentos de falta de TEEs com estado externo,
evitando a utilização de protocolos BFT, que incorrem num custo
superior. Mostramos que protocolos CFT existentes podem ser
adaptados facilmente para este modelo de faltas com poucas
alterações, mantendo a sua performance original.
\vskip -0cm
Para ilustrar a utilidade da replicação de TEEs segundo o
modelo RR, construímos um serviço de metadados replicado chamado
TEEMS, que pode ser usado para acrescentar confidencialidade,
integridade e frescura a armazenamento em cloud ao nível das
proteções dos TEEs. Seguidamente, exemplificamos a generalidade
do modelo RR, aplicando-o ao contexto de KVSs replicadas. KVSs
replicadas atuais escolhem entre latência, \textit{throughput} e
durabilidade, dependendo se sincronizam as escritas antes de
responderem aos pedidos (o que oferece uma \textit{performance}
fraca) ou se combinam várias escritas numa \textit{batch} em
memória (possivelmente perdendo dados se os nós forem abaixo
antes da escrita ser sincronizada). Observamos que o último
caso effetivamente corresponde a uma reversão de estado, que é
capturada pelo modelo RR.
\vskip 0cm
Impelmentámos os vários protocolos e sistemas, e a nossa
avaliação demonstra que a \textit{performance} é comparável à de
sistemas CFT mas com garantias mais fortes e significativamente
superior à de sistemas BFT.
